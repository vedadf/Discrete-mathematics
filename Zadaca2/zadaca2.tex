\documentclass[12pt]{article}

\usepackage{amsmath}
\usepackage{amssymb}
\usepackage{graphicx}
\usepackage{pifont}
\usepackage{listings}
\usepackage{xcolor}
\usepackage[T1]{fontenc}
\usepackage[utf8]{inputenc}
\usepackage{textcomp}
\usepackage{booktabs}

\definecolor{listinggray}{gray}{0.9}
\definecolor{lbcolor}{rgb}{0.9,0.9,0.9}
\lstset{
	language=C++,
	frame = tb,
	numbers = left,
	showstringspaces = false,
	basicstyle=\ttfamily,
	keywordstyle=\color{blue}\ttfamily,
	stringstyle=\color{red}\ttfamily,
	commentstyle=\color{green}\ttfamily,
	morecomment=[l][\color{magenta}]{\#} 
}

\begin{document}

\begin{titlepage}
	\newcommand{\HRule}{\rule{\linewidth}{0.5mm}}
	
	\center
	
	\textsc{\Large Elektrotehnički fakultet Univerziteta Sarajevo}\\[4cm]
	
	{\huge\bfseries Diskretna Matematika\vspace{5mm}

 	Zadaća 2}\\[4.5cm]

	\begin{minipage}{0.4\textwidth}
		\begin{flushleft}
			\large
			\textit{Student}\\
			Vedad Fejzagić\\[5mm]
			\textit{Broj indeksa}\\
			17336\\[5mm]
			\textit{Grupa}\\
			RI2-2
		\end{flushleft}
	\end{minipage}
	~
	\begin{minipage}{0.4\textwidth}
		\begin{flushright}
			\large
			\textbf{Demonstrator}\\
			\hspace{10mm}Šeila Bečirović
		\end{flushright}
	\end{minipage}
	
	\vfill\vfill\vfill
	
	{\large\today}
	
	\vfill
	
\end{titlepage}


\newpage
\section*{Zadatak 1\label{Z1}}

\underline{Postavka:}

Odredite na koliko je različitih načina moguće razmjestiti 7 studenata i 3 profesora oko okruglog stola ako je poznato da se profesori međusobno ne podnose i ne mogu sjediti jedan do drugog.

\underline{Rješenje:}\\

Sedam studenata možemo raspodijeliti oko okruglog stola na $6! = 6  \cdot 5 \cdot 4 \cdot 3 \cdot 2 \cdot 1$ načina. Profesore raspoređujemo između svaka dva studenta. Prvog profesora možemo na $7$ mjesta smjestiti, drugog na 6 i trećeg na 5 mjesta. Dakle $7 \cdot 6 \cdot 5$ načina.

Na osnovu multiplikativnog principa, ukupan broj načina je:

$$\frac{P_{7}}{7} \cdot 7^{(3)} = 6! \cdot 7 \cdot 6 \cdot 5 = \underline{151200}$$
\newpage
\section*{Zadatak 2\label{Z2}}

\underline{Postavka:}

Potrebno je formirati šestočlanu ekipu za međunarodno softversko-hardversko takmičenje. Uvjeti su da ekipa mora imati barem tri studenta sa smjera RI, dok su studenti drugih smjerova poželjni (zbog većeg hardverskog znanja) ali ne i obavezni. Za takmičenje se prijavilo 5 studenata smjera RI i 6 studenata smjera AiE (dok studenti drugih smjerova nisu bili zainteresirani). Odredite na koliko načina je moguće odabrati traženu ekipu. Koliko će iznositi broj mogućih ekipa ukoliko se postavi dodatno ograničenje da ekipa mora imati i barem dva studenta smjera AiE?

\underline{Rješenje:}\\

Potrebno je izabrati minimalno k studenata sa smjera RI i 6-k studenata sa smjera AiE, gdje je $k \geq 3$. Dakle, problem se svodi na računanje sume $\sum_{k=3}^{5} C_{5}^k \cdot C_{6}^{6-k}$. Dakle:

$$C_{5}^3 \cdot C_{6}^3 + C_{5}^4 \cdot C_{6}^2 +  C_{5}^5 \cdot C_{6}^1 = 10 \cdot 20 + 5 \cdot 15 + 6 = \underline{281}$$

Dodatno ograničenje: barem 2 studenta AiE:

$$C_{5}^3 \cdot C_{6}^3 + C_{5}^4 \cdot C_{6}^2 = 10 \cdot 20 + 5 \cdot 15 = \underline{275}$$
\newpage
\section*{Zadatak 3\label{Z3}}

\underline{Postavka:}

Odredite na koliko načina možemo raspodijeliti 10 jabuka, 9 banana i 11 kajsija među troje djece, pri čemu se pretpostavlja se da se primjerci iste voćke ne mogu međusobno razlikovati (npr. sve jabuke su iste).

\underline{Rješenje:}\\

Podijelit ćemo prvo jabuke, pa kruške pa kajsije. Pošto su podjele neovisne, koristimo multiplikativni princip:

$$\overline{C}_{3}^{10} \cdot \overline{C}_{3}^9 \cdot \overline{C}_{3}^{11} = C_{12}^{10} \cdot C_{11}^9 \cdot C_{13}^{11} = \dbinom{12}{10} \dbinom{11}{9} \dbinom{13}{11} = \underline{283140}$$
\newpage
\section*{Zadatak 4\label{Z4}}	 

\underline{Postavka:}

Na stolu se nalazi određena količina papirića, pri čemu se na svakom od papirića nalazi po jedno slovo. Na 3 papirića se nalazi slovo Z, na 4 papirića se nalazi slovo D, na 5 papirića slovo P i na 4 papirića slovo J. Odredite koliko se različitih šestoslovnih riječi može napisati slažući uzete papiriće jedan do drugog (nebitno je imaju li te riječi smisla ili ne).

\underline{Rješenje:}\\

Imamo \{$3\cdot Z, 4 \cdot D, 5\cdot P, 4\cdot J$\}. Potrebno je izračunati $\overline{P}_{4; 3, 4, 5, 4}^{6}$. To ćemo uraditi preko funkcije generatrise:

$$\psi_{n; m_{1}, m_{2},..., m_{n}}(t) = \prod_{i=1}^{n}\sum_{j=0}^{m_{i}} \frac{t^j}{j!}$$\\
$$\psi_{4; 3, 4, 5, 4}(t) = \bigg(1 + t + \frac{t^2}{2} + \frac{t^3}{6}\bigg)\bigg(1 + t + \frac{t^2}{2} + \frac{t^3}{6} + \frac{t^4}{24}\bigg)^2\bigg(1 + t + \frac{t^2}{2} + \frac{t^3}{6} + \frac{t^4}{24} + \frac{t^5}{120}\bigg)$$\\
$$\psi_{4; 3, 4, 5, 4}(t) =  1 + 4t +  8 t^2 +  \frac{32}{3}t^3 + \frac{85}{8} t^4 +  \frac{503}{60}t^5+\frac{1301}{240} t^6 + ...$$\\

Koeficijent uz $t^6$ je $\frac{1301}{240}$, dakle rješenje je:

$$\frac{1301}{240} \cdot 6! = \underline{3903}$$
\newpage
\section*{Zadatak 5\label{Z5}}	 

\underline{Postavka:}

Odredite koliko se različitih paketa koji sadrže 5 voćki može napraviti ukoliko nam je raspolaganju 6 krušaka, 3 kajsije, 3 banane, 3 jabuke i 1 naranča (pri čemu se pretpostavlja da ne pravimo razliku između primjeraka iste voćke).

\underline{Rješenje:}\\

Radi se o 5-kombinacija multiskupa $ \{5 \cdot kruška, 3\cdot kajsija, 3\cdot banana, 3\cdot jabuka, 1\cdot naranča\}$, takod a je broj traženih paketa zapravo $\overline{C}_{5; 6, 3, 3, 3, 1}^{5}$. Rješavamo pomoču funkcije generatrise za koju vrijedi:

$$\varphi_{n; m_{1}, m_{2},..., m_{n}}(t) = \prod_{i=1}^{n} \sum_{j=0}^{m_{i}} t^j$$\\
$$\varphi_{5; 6,3,3,3,1}(t) = (1 + t + t^2 + t^3 + t^4 + t^5 + t^6)(1 + t + t^2 + t^3)^3(1 + t)$$
$$\varphi_{5; 6,3,3,3,1}(t) = 1 + 5t + 14t^2 + 27t^3 + 40t^4 + 49t^5 + ...$$\\

Koeficijent uz $t^5$ je $\underline{49}$ što je zapravo rješenje zadatka.
\newpage
\section*{Zadatak 6\label{Z6}}	 

\underline{Postavka:}

Odredite na koliko načina se može rasporediti 9 identičnih kuglica u 4 različitih kutija, ali tako da u svakoj kutiji bude najviše 4 kuglica.

\underline{Rješenje:}\\

Potrebno je odrediti broj kombinacija sa ponavljanjem klase 9 skupa od 4 elementa u kojoj se svaki element javlja najviše 4 puta. Dakle zanima nas $\overline{C}_{4; S,S,S,S}^{9}$. Koristimo funkcije generatrise:

$$\varphi_{4; S,S,S,S}(t) = \prod_{i=1}^4 \sum_{j \in S} t^j = (1 + t + t^2 + t^3 + t^4)^4$$\\

Izrazimo sumu ovog geometrijskog reda na drugi način:

$$1 + t + t^2 + t^3 + t^4 = s / \cdot t$$
$$t + t^2 + t^3 + t^4 + t^5 = st / (+1)$$
$$1 + t + t^2 + t^3 + t^4 + t^5 = st + 1 $$
$$s + t^5 = st + 1$$
$$s = \frac{1-t^5}{1-t}$$

Pa je:

$$\varphi_{4; S,S,S,S}(t) = \bigg(\frac{1-t^5}{1-t}\bigg)^4 = (1 - t^5)^4 (1 - t)^{-4} = \bigg(\sum_{i=0}^{4} \dbinom{4}{i} (-t^5)^i \bigg) \bigg(\sum_{i=0}^{\infty} \dbinom{-4}{i} (-t)^i \bigg)$$

Raspisivanjem ovih suma, zatim množenjem članova čiji produkt daje koeficijent uz $t^9$, te izračunavanjem tako dobijenog izraza, dobije se rezultat:

$$\overline{C}_{4; S,S,S,S}^{9} = 80$$

Alternativno, mogli smo množiti izraz grubom silom i tako dobiti isto rješenje.
\newpage
\section*{Zadatak 7\label{Z7}}	 

\underline{Postavka:}

Odredite na koliko načina se 13 različitih predmeta upakovati u 7 identičnih vreća (koje nemaju nikakav identitet po kojem bi se mogle razlikovati), pri čemu se dopušta i da neke od vreća ostanu prazne.

\underline{Rješenje:}\\

Ovdje se radi o Stirlingovim brojevima druge vrste, pri čemu je $n = 13$ i $k = 7$, te ćemo razmatrati više slučajeva:

\begin{center}
1. Nijedna vreća nije prazna

2. Jedna vreća prazna 

3. Dvije vreće prazne

4. Tri vreće prazne

5. Četiri vreće prazne

6. Pet vreća praznih

7. Šest vreća praznih
\end{center}

I na kraju sabrati sve slučajeve. Primjetimo da se suma svih slučajeva može napisati kao suma Stirlingovih brojeva druge vrste $\sum_{i=1}^7 S_{13}^i$. Rješavamo pomoću tabele:

\begin{table}[]
\centering
\caption{Stirlingovi brojevi druge vrste}
\label{Tabela1}
\begin{tabular}{@{}cccccccc@{}}
1 & 0 & 0    & 0      & 0       & 0       & 0       & 0       \\
0 & 1 & 0    & 0      & 0       & 0       & 0       & 0       \\
0 & 1 & 1    & 0      & 0       & 0       & 0       & 0       \\
0 & 1 & 3    & 1      & 0       & 0       & 0       & 0       \\
0 & 1 & 7    & 6      & 1       & 0       & 0       & 0       \\
0 & 1 & 15   & 25     & 10      & 1       & 0       & 0       \\
0 & 1 & 31   & 90     & 65      & 15      & 1       & 0       \\
0 & 1 & 63   & 301    & 350     & 140     & 21      & 1       \\
0 & 1 & 127  & 966    & 1701    & 1050    & 266     & 28      \\
0 & 1 & 255  & 3025   & 7770    & 6951    & 2646    & 462     \\
0 & 1 & 511  & 9330   & 34105   & 42525   & 22827   & 5880    \\
0 & 1 & 1023 & 28501  & 145750  & 246730  & 179487  & 63987   \\
0 & 1 & 2047 & 86526  & 611501  & 1379400 & 1323652 & 627396  \\
0 & 1 & 4095 & 261625 & 2532530 & 7508501 & 9321312 & 5715424
\end{tabular}
\end{table}
\newpage
Sada samo saberemo sve brojeve u svim kolonama u zadnjem redu, dakle:

$$\sum_{i=1}^7 S_{13}^i = 5715424 + 9321312 +  7508501 + 2532530 + 261625 + 4095 + 1 = \underline{25343488}$$

\newpage
\section*{Zadatak 8\label{Z8}}	 

\underline{Postavka:}

Odredite na koliko se načina može 11 kamenčića razvrstati u 6 gomilica. Pri tome se i kamenčići i gomilice smatraju identičnim (odnosno ni kamenčići ni gomilice nemaju nikakav identitet po kojem bi se mogli razlikovati).

\underline{Rješenje:}\\

\newpage
\section*{Zadatak 9\label{Z9}}	 

\underline{Postavka:}

Odredite na koliko načina se broj 13 može rastaviti na sabirke koji su prirodni brojevi, pri čemu njihov poredak nije bitan, ali pod dodatnim uvjetom da se sabirak 1 smije pojaviti najviše 2 puta, dok se sabirak 3 smije pojaviti samo neparan broj puta.

\underline{Rješenje:}\\
\end{document}