\documentclass[12pt]{article}

\usepackage{amsmath}
\usepackage{amssymb}
\usepackage{graphicx}
\usepackage{pifont}
\usepackage{listings}
\usepackage{xcolor}
\usepackage[T1]{fontenc}
\usepackage[utf8]{inputenc}
\usepackage{textcomp}
\usepackage{booktabs}
\usepackage{multirow}

\usepackage{pgf}
\usepackage{tikz}
\usetikzlibrary{arrows,automata}

\definecolor{listinggray}{gray}{0.9}
\definecolor{lbcolor}{rgb}{0.9,0.9,0.9}
\lstset{
	language=C++,
	frame = tb,
	numbers = left,
	showstringspaces = false,
	basicstyle=\ttfamily,
	keywordstyle=\color{blue}\ttfamily,
	stringstyle=\color{red}\ttfamily,
	commentstyle=\color{green}\ttfamily,
	morecomment=[l][\color{magenta}]{\#} 
}

\begin{document}

\begin{titlepage}
	\newcommand{\HRule}{\rule{\linewidth}{0.5mm}}
	
	\center
	
	\textsc{\Large Elektrotehnički fakultet Univerziteta Sarajevo}\\[4cm]
	
	{\huge\bfseries Diskretna Matematika\vspace{5mm}

 	Zadaća 4}\\[4.5cm]

	\begin{minipage}{0.4\textwidth}
		\begin{flushleft}
			\large
			\textit{Student}\\
			Vedad Fejzagić\\[5mm]
			\textit{Broj indeksa}\\
			17336\\[5mm]
			\textit{Grupa}\\
			RI2-2
		\end{flushleft}
	\end{minipage}
	~
	\begin{minipage}{0.4\textwidth}
		\begin{flushright}
			\large
			\textbf{Demonstrator}\\
			\hspace{10mm}Šeila Bečirović
		\end{flushright}
	\end{minipage}
	
	\vfill\vfill\vfill
	
	{\large\today}
	
	\vfill
	
\end{titlepage}


\newpage
\section*{Zadatak 1\label{Z1}}

\underline{Postavka:}

\begin{tabular}{ | c || c  | c | c | c | c | c | c | c | }
\hline
 & Amcazo & Brot & Quwuti & Sosyab & Uhsuru & Urasoto & Xanu & Zixa\\
 \hline
 \hline
Amcazo & 0 & 500 & 260 & 1030 & 470 & - & 1210 & 220\\
 \hline
Brot & 500 & 0 & 1370 & - & 610 & 1060 & 640 & 610\\
 \hline
Quwuti & 260 & 1370 & 0 & - & 510 & 460 & - & 600\\
 \hline
Sosyab & 1030 & - & - & 0 & 480 & 250 & 770 & 500\\
 \hline
Uhsuru & 470 & 610 & 510 & 480 & 0 & 1490 & 270 & 260\\
 \hline
Urasoto & - & 1060 & 460 & 250 & 1490 & 0 & 370 & 240\\
 \hline
Xanu & 1210 & 640 & - & 770 & 270 & 370 & 0 & 370\\
 \hline
Zixa & 220 & 610 & 600 & 500 & 260 & 240 & 370 & 0\\
 \hline
\end{tabular}

Rjesenje:

\begin{tabular}{ | c || c  | c | c | c | c | c | c | c | }
\hline
 & Amcazo & Brot & Quwuti & Sosyab & Uhsuru & Urasoto & Xanu & Zixa\\
 \hline
 \hline
Amcazo & 0 & 500 & 260 & 710 & 470 & 460 & 590 & 220\\
 \hline
Brot & 500 & 0 & 760 & 1090 & 610 & 850 & 640 & 610\\
 \hline
Quwuti & 260 & 760 & 0 & 710 & 510 & 460 & 780 & 480\\
 \hline
Sosyab & 710 & 1090 & 710 & 0 & 480 & 250 & 620 & 490\\
 \hline
Uhsuru & 470 & 610 & 510 & 480 & 0 & 500 & 270 & 260\\
 \hline
Urasoto & 460 & 850 & 460 & 250 & 500 & 0 & 370 & 240\\
 \hline
Xanu & 590 & 640 & 780 & 620 & 270 & 370 & 0 & 370\\
 \hline
Zixa & 220 & 610 & 480 & 490 & 260 & 240 & 370 & 0\\
 \hline
\end{tabular}




\end{document}