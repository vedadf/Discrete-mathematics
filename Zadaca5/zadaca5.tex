\documentclass[12pt]{article}

\usepackage{amsmath}
\usepackage{amssymb}
\usepackage{graphicx}
\usepackage{pifont}
\usepackage{listings}
\usepackage{xcolor}
\usepackage[T1]{fontenc}
\usepackage[utf8]{inputenc}
\usepackage{textcomp}
\usepackage{booktabs}
\usepackage{multirow}
\usepackage{mathtools}
\DeclarePairedDelimiter\floor{\lfloor}{\rfloor}
\DeclarePairedDelimiter\abs{\lvert}{\rvert}%
\makeatletter
\let\oldabs\abs
\def\abs{\@ifstar{\oldabs}{\oldabs*}}
\usepackage{pgf}
\usepackage{tikz}
\usetikzlibrary{arrows,automata}

\definecolor{listinggray}{gray}{0.9}
\definecolor{lbcolor}{rgb}{0.9,0.9,0.9}
\lstset{
	language=C++,
	frame = tb,
	numbers = left,
	showstringspaces = false,
	basicstyle=\ttfamily,
	keywordstyle=\color{blue}\ttfamily,
	stringstyle=\color{red}\ttfamily,
	commentstyle=\color{green}\ttfamily,
	morecomment=[l][\color{magenta}]{\#} 
}

\begin{document}

\begin{titlepage}
	\newcommand{\HRule}{\rule{\linewidth}{0.5mm}}
	
	\center
	
	\textsc{\Large Elektrotehnički fakultet Univerziteta Sarajevo}\\[4cm]
	
	{\huge\bfseries Diskretna Matematika\vspace{5mm}

 	Zadaća 5}\\[4.5cm]

	\begin{minipage}{0.4\textwidth}
		\begin{flushleft}
			\large
			\textit{Student}\\
			Vedad Fejzagić\\[5mm]
			\textit{Broj indeksa}\\
			17336\\[5mm]
			\textit{Grupa}\\
			RI2-2
		\end{flushleft}
	\end{minipage}
	~
	\begin{minipage}{0.4\textwidth}
		\begin{flushright}
			\large
			\textbf{Demonstrator}\\
			\hspace{10mm}Šeila Bečirović
		\end{flushright}
	\end{minipage}
	
	\vfill\vfill\vfill
	
	{\large\today}
	
	\vfill
	
\end{titlepage}

\newpage

\section*{Zadatak 1\label{Z1}}

\underline{Postavka:}

Dat je periodični diskretni signal perioda 6, čije vrijednosti u trenucima 0, 1, 2, 3, 4 i 5 respektivno iznose 4, –6, 4, 0, 5 i –8.


Predstavite ovaj diskretni signal formulom u kojoj se javlja cijeli dio broja;

Predstavite ovaj signal diskretnim Fourierovim redom;

Odredite amplitudni i fazni spektar za ovaj signal i predstavite ga u vidu sume harmonika.


\underline{Rješenje:}

$$N = 6$$
$$n = 0, 1, 2, 3, 4, 5$$
$$x_0 = 4, x_1 = -6, x_2 = 4, x_3 = 0, x_4 = 5, x_5 = -8$$

a)

Predstavljamo diskretni signal formulom u kojoj se javlja cijeli dio broja:

$$x_n = x_{-1} + (x_0 - x_{-1}) \floor[\Big]{\frac{n}{6}} + (x_1 - x_0) \floor[\Big]{\frac{n - 1}{6}} + (x_2 - x_1)  \floor[\Big]{\frac{n - 2}{6}} + $$
$$ + (x_3 - x_2)  \floor[\Big]{\frac{n - 3}{6}} + (x_4 - x_3)  \floor[\Big]{\frac{n - 4}{6}}  + (x_5 - x_4)  \floor[\Big]{\frac{n - 5}{6}}$$ 

Pa je rješenje:

$$x_n = -8 + 12 \floor[\Big]{\frac{n}{6}} - 10 \floor[\Big]{\frac{n - 1}{6}} + 10  \floor[\Big]{\frac{n - 2}{6}} - $$
$$ - 4\floor[\Big]{\frac{n - 3}{6}} + 5  \floor[\Big]{\frac{n - 4}{6}}  - 13 \floor[\Big]{\frac{n - 5}{6}}$$
\newpage
b) 

Računamo $a_k, b_k$ te na kraju $x_n$. $N = 6$ je parno pa $b_3 = 0$.

$$a_0 = \frac{2}{6} \sum_{n = 0}^{5} x_n \cdot cos 0 = \frac{2}{6} (4 - 6 + 4 + 5 - 8) = - \frac{1}{3}$$

Na isti način računamo ostale koeficijente, pa se dobije:

$$a_1 = - \frac{5}{2}, a_2 = \frac{13}{6}, a_3 = \frac{27}{6}$$

$$b_1 = \frac{\sqrt{3}}{6}, b_2 = \frac{\sqrt{3}}{2}$$

Na kraju, razvoj u Furijerov red:

$$x_n = - \frac{1}{6} + (- \frac{5}{2} cos \frac{\pi n}{3} + \frac{\sqrt{3}}{6} sin \frac{\pi n}{3}) + (\frac{13}{6}cos\frac{2 \pi n}{3} + \frac{\sqrt{3}}{2} sin \frac{2\pi n}{3}) + (\frac{27}{6} cos \pi n)$$
 
c)

$$A_0 = - \frac{1}{6}$$
$$\varphi_{0} = 0$$

\newpage

\section*{Zadatak 2\label{Z2}}

\underline{Postavka:}
Ispitajte koji su od sljedećih diskretnih signala periodični a koji nisu. Za one koji jesu, nađite osnovni period N:

$$2 sin (7n \pi + \pi/5)$$
$$3 + cos2 (12n\pi/5 – 1)$$
$$(–1)n cos (4n\pi/7)$$
$$sin (2n \pi/3)$$

Odgovori moraju biti obrazloženi, inače se neće priznavati!


\underline{Rješenje:}
\\
Diskretni harmonijski signali su periodični samo ako je moguće naći prirodan broj k takav da je $\frac{2 k \pi}{\Omega}$ također prirodan broj. Ovo svojstvo ćemo iskoristiti u podzadacima a) i b). Za c) i d) ćemo primjeniti svojstvo da je diskretni signal periodičan ako postoji cijeli broj $N \neq 0$  takav da za svako $n \in Z$ vrijedi $x_{n + N} = x_n$, pri tome broj N je period diskretnog signala.
\\
a)

Iz postavke odredimo omega, pa uvrstimo u formulu $\frac{2 k \pi}{\Omega}$.

$$\Omega = 7 \pi \to \frac{2 k \pi}{7 \pi} = \frac{2k}{7}$$

Ovaj izraz je najrpije zadovoljen ako je $k = 7$. Uvrštavanjem se dobije: $N = 2$. Dakle ovaj diskretni signal je periodičan sa osnovnim periodom $N = 2$.
\newpage
b)

$$\Omega = \frac{12 \pi}{5} \to \frac{10 k}{12} = \frac{5k}{6}$$

Ovaj izraz je najrpije zadovoljen ako je $k = 6$. Uvrštavanjem se dobije: $N = 5$. Dakle ovaj diskretni signal je periodičan sa osnovnim periodom $N = 5$.

c)

Primjenjujemo pravilo $x_{n + N} = x_n$:

$$(-1) ^{n + N} cos \frac{4(n + N) \pi}{7} = (-1)^n (-1) ^N (cos \frac{4 n \pi}{7} cos \frac{4 N \pi}{7} - sin \frac{4 n \pi}{7} sin \frac{
 4 N \pi}{7})$$

N mora biti parno i mora vrijediti:

$$cos \frac{4 N \pi}{7} = 1 \land sin \frac{4 N \pi}{7} = 0$$

Ovo je ispunjeno za: 

$$\frac{4 N \pi}{7} = 2 k \pi$$
$$\frac{2N}{7} = k$$
i N parno.

Za N = 14, ova jednakost je zadovoljena.  Dakle ovaj diskretni signal je periodičan sa osnovnim periodom $N = 14$.

d)

Primjenjujemo pravilo $x_{n + N} = x_n$:

Dobije se $sin \frac{2^n \cdot 2^N \pi}{3}$

Da bi dobijeni izraz bio jednak $x_n$, mora vrijediti N = 0. Dakle, ovaj diskretni signal nije periodičan.

\newpage

\section*{Zadatak 3\label{Z3}}

\underline{Postavka:}

Odredite i nacrtajte amplitudno-frekventnu i fazno-frekventnu karakteristiku diskretnog sistema opisanog diferentnom jednačinom $y_n + 6y_{n–1} = –6x_n + 7 x_{n–1}$, a nakon toga, odredite odziv ovog sistema na periodični signal iz Zadatka 1. Uputa: razmotrite prolazak svakog od harmonika posebno.


\underline{Rješenje:}
\\
Prvo računamo funkciju sistema $H(z)$. Izjednačavamo, $x_n = z^n$, $y_n  = z^n H(z)$. Dobijena diferentna jednačine postaje:

$$z^n H(z) + 6 z^{n - 1} H(z) = - 6 z^n + 7 z^{n - 1}$$

Slijedi:

$$H(z) = \frac{  - 6 z^n + 7 z^{n - 1} }{ z^n + 6 z^{n - 1}  } = \frac{-6 + 7 z^{-1}}{1 + 6z^{-1}} = \frac{-6 z + 7}{z + 6}$$

Računamo amplitudno-frekventnu karakteristiku:

$$A(\Omega) = |H(e ^ {i \Omega})| = \abs{\frac{-6 e ^{i \Omega} + 7}{e ^{i \Omega} + 6}} = \frac{-6 cos \Omega + 7 - 6 i sin\Omega}{cos \Omega + 6 + i sin \Omega} = \sqrt{\frac{(-6 cos\Omega + 7)^2 + 36 sin^2 \Omega}{(cos \Omega + 6)^2 - sin^2 \Omega}}$$

$$A(\Omega) = \sqrt{\frac{-84 cos \Omega + 85}{2 cos ^2 \Omega + 12 cos \Omega + 36}}$$

Računamo fazno-frekventnu karakteristiku:

Sličan postupak kao i kod računanja amplitudne-frekventne karakteristike, te odvajanje realnog i kompleksnog koeficijenta u zasebne sabrike za nalaženje arg:

$$A(\varphi) = arg H(e ^{i\Omega}) = arg \frac{-6 cos\Omega + 7 - 6 i sin \Omega}{ cos \Omega + 6 + i sin \Omega} \cdot \frac{cos\Omega + 6 - i sin \Omega}{cos\Omega + 6 - i sin\Omega}$$
$$A(\varphi) = arg \frac{-29 cos \Omega + 36 - i \cdot 43 sin \Omega}{(cos\Omega + 6 + i sin\Omega)(cos\Omega + 6 - i sin\Omega)}$$

Realni dio je uvijek $Re(z) > 0$, pa možemo pisati:

$$A(\varphi) = arctg(- \frac{43 sin \Omega}{36 - 29cos\Omega})$$

Odziv ovog sistema na periodični signal iz Zadatka 1:

$$y_{n - 1} = 0$$

$$y_0 = -6 \cdot 4 + 7 \cdot (-8) - 0 = - 80$$

Na isti način računamo ostale:

$$y_1 = 544, y_2 = 3330, y_3 = -19952, y_4 = 119682, y_5 = -718079$$

\newpage

\section*{Zadatak 4\label{Z4}}

\underline{Postavka:}

Nađite z-transformaciju sekvence $x_n = (n^2 cos (n \pi / 6) + 4^n / (2_n + 1)! ) u_{n–3}$. Pri tome se po potrebi možete koristiti tablicama i osobinama z-transformacije.

\underline{Rješenje:}

Ovo je sekvenca oblika $y_n u_{n-3}$. Potrebno je izvesti pravilo za $Z\{y_n, u_{n - k}\}$. Ako stavimo da je $y_n = w_{n - k}$, tada je $w_n = y_{n + k}$, stoga imamo:

$$Z\{y_n u_{n - k}\} = Z\{w_{n-k} u_{n - k}\} = Y(z) - \sum_{i = 0}^{k - 1} y_i z^{-i}$$

Korištenjem tablice transformacije imamo:

$$Z\{cos \frac{n \pi}{6}\} = \frac{z (z - \frac{1}{2})}{z^2 - z + 1}$$

$$Z\{n^2 \cdot cos \frac{n \pi}{6}\} = z^2 (\frac{-z (z - \frac{1}{2})}{z^2 - z + 1}) ' = z^2 \frac{\frac{1}{2} z^2 + \frac{1}{2} - 2z}{(z^2 - z + 1)^2}$$

Računamo drugu z - transformaciju:

Imamo da je $Z\{\frac{1}{n!}\} = e ^{\frac{1}{z}}$

Koristimo osobinu: $Z\{y_n\} = y(z) \implies Z\{y_{2n + 1}\} = \frac{Y(\sqrt{z}) - Y(- \sqrt{- z})}{2 \sqrt{z}}$
Pa je

$$Z\{ \frac{1}{(2n + 1)!} \} = \frac{e^{\frac{2}{\sqrt{z}}} - 1}{2 e ^{\frac{2}{\sqrt{z}}}}$$

$$Z\{ \frac{4^n}{(2n + 1)!}\} = \frac{  e^{ - \frac{2 \sqrt{z}}{z} } - 1  }{2 e^{ \frac{2 \sqrt{z}}{z}}}$$

Na kraju:

$$Z \{x_n\} = Y(z) - y_0 - y_1 z^{-1} - y_2 z^{-2} = Z\{n^2 \cdot cos \frac{n \pi}{6}\} + Z\{ \frac{4^n}{(2n + 1)!}\} - y_0 - y_1 z^{-1} - y_2 z^{-2}$$

$$Z \{x_n\} = z^2 \frac{\frac{1}{2} z^2 + \frac{1}{2} - 2z}{(z^2 - z + 1)^2} + \frac{  e^{ - \frac{2 \sqrt{z}}{z} } - 1  }{2 e^{ \frac{2 \sqrt{z}}{z}}} - 1 - \frac{1}{z} (\frac{\sqrt{3}}{2} + \frac{2}{3}) - \frac{1}{z^2} (\frac{1}{2} + \frac{16}{120})$$

\newpage

\section*{Zadatak 5\label{Z5}}

\underline{Postavka:}

Dat je diskretni sistem opisan diferentnom jednačinom $2 y_{n+1} + 7 y_n = 3 x_{n+1} – 5 x_{n–2}$. Nađite odziv ovog sistema na pobudu $x_n = n cos (2n \pi / 3) u_n$. Rješenje treba izraziti u obliku u kojem se ne javljaju kompleksni brojevi.

\underline{Rješenje:}

Najprije računamo funkciju sistema:

$$x_n = z^n, y_n = z^n H(z)$$

Dobije se:

$$H (z) = \frac{3z^3 - 5}{z^2 (2z + 7)}$$

Dalje, računamo impulsni odziv $h_n = Z^{-1}(H(z))$

$$P(z) = 3z^3 - 5$$
$$Q(z) = z^2(2z + 7)$$

Nule polinama Q(z) su: z = 0 čija je višestrukost 2 i $z = - \frac{7}{2}$





















\end{document}