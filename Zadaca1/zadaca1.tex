\documentclass[12pt]{article}

\usepackage{amsmath}
\usepackage{amssymb}
\usepackage{graphicx}
\usepackage{pifont}
\usepackage[T1]{fontenc}
\usepackage[utf8]{inputenc}

\title{
  Diskretna matematika\\
  \large Zadaća 1 \\}
\author{Vedad Fejzagić}
\date{Oktobar 22, 2017}

\begin{document}

\maketitle

\iffalse

\section{Introduction}
\subsection{Who am I}

Hello world!
\textbf{This is bold text}
\textit{This is italic.}


\emph{this is emphatic.}
\underline{Some underline.}

``Propper quotes''

\section{List}
\begin{enumerate}
\item Bread
\item butter
\end{enumerate}

\section{Unordered List}
\begin{itemize}
\item Bread
	\begin{itemize}
	\item Bread
	\end{itemize}
\item butter
\end{itemize}
Ovo je \ref{Z1}
\fi

\newpage

\section*{Zadatak 1\label{Z1}}

\hspace{0.65cm}Ako označimo tablete T1, T2 i T3 kao x, y i z respektivno, problem svodimo na rješavanje sljedeće diofantove jednačine sa 3 nepoznate:

$$15x + 33y + 27z = 162$$\\

Očigledno je da vrijedi:

$$NZD(15, 33, 27) = 3$$\\

Dokažimo koristeći Euklidov algoritam:

$$NZD(15, 33, 27) = NZD(NZD(15, 33), 27) =$$ 
$$= NZD(3, 27) = NZD(27, 3) = 3$$\\

Dalje, s obzirom da je $NZD(15, 33, 27) = 3, 3 \mid 162$, zadana diofantova jednačina je rješiva.

$$15x + 33y + 27z = 162$$
$$5x + 11y + 9z = 54$$
$$5x + 11y = 54 - 9z$$\\

Pošto je $NZD(5, 11) = 1$, rješenja za $x$ i $y$ će postojati akko je $1 \mid (54 - 9z)$ tj. ako postoji $k\in Z$ takav da vrijedi $54 - 9z = k$. Ovo je diofantova jednačina, dakle $NZD(9, 1) = 1, 1 \mid 54$, te je potrebno izraziti $NZD(9, 1) = 1$ kao linearnu kombinaciju 9 i 1:

$$9 = 1 \cdot 8 + 1 \implies 1 = 9 - 1 \cdot 8$$\\

Jedno rješenje je:

$$z^{*} = 54$$
$$k^{*} = - 8 \cdot 54 = - 432$$\\

Opće rješenje za $z$($k$ nas ne interesuje za konkretan problem):

$$z = 54 + t, t\in Z$$\\

Vraćamo u početnu jednačinu:

$$5x + 11y = 54 - 9(54 + t)$$
$$5x + 11y = -432 - 9t$$\\

Dobivena jednačina je diofantova. Očigledno je $NZD(15, 11) = 1$, potrebno je izraziti $NZD(15, 11) = 1$ preko linearne kombinacije 15 i 11:

$$11 = 2 \cdot 5 + 1 \implies 1 = 11 - 2 \cdot 5 = - 2 \cdot 5 + 11$$\\

Pa su opća rješenja:

$$x = 864 + 18t + 11s$$
$$y = -432 - 9t - 5s$$
$$z = 54 + t$$
$$t, s\in Z$$
\begin{center}
uz ograničenja $x, y, z > 0$
\end{center}

Pristupamo rješavanju sistema nejednačina:

\[
x = 864 + 18t + 11s > 0 \label{eq:gen1} \tag{1}
\]
\[
y = -432 - 9t - 5s \implies y = 432 + 9t - 5s < 0 \label{eq:gen2} \tag{2}
\]
$$z = 54 + t > 0 \implies t > - 54$$

Iz (\ref{eq:gen2}):

\[
s < \frac{- 9t - 432}{5} \label{eq:gens1} \tag{A}
\]

Iz (\ref{eq:gen1}):

\[
s > \frac{- 864 - 18t}{11} \label{eq:gens2} \tag{B}
\]\\

Možemo zaključiti:

$$(\ref{eq:gen1}) \land (\ref{eq:gen2}) \implies \frac{- 9t - 432}{5} > s > \frac{- 864 - 18t}{11} \implies$$
$$\implies (- 9t - 432) \cdot 11 > (- 864 - 18t) \cdot 5$$\
$$- 9t - 432 > 0$$\
$$t < \frac{- 432}{9}$$\
$$t < -48$$\\

Rješenja za t: 
$$(t > -54) \land (t < -48) \implies t\in (-48, -54)$$

tj. 

$$t\in [-49, -53], t\in Z$$\\

Dalje, računamo vrijednost s, $\forall t\in [-49, -53] \land t\in Z$ koristeći nejednakosti \ref{eq:gens1} i \ref{eq:gens2}. Lahko se pokaže da vrijednosti t = - 49, t = - 50 i t = - 53 ne daju vrijednost $s \in Z$, dakle te vrijednosti odbacujemo.\\

\underline{Za t = - 51:}

$$(\ref{eq:gens1}) \implies s < \frac{27}{5} ( = 5.4)$$
$$(\ref{eq:gens2}) \implies s > \frac{54}{11} ( \sim 4.9)$$
$$s \in (\frac{54}{11}, \frac{27}{5})$$\\

Pa jedina vrijednost u skupu $Z$ na dobivenom intervalu je $s = 5$. Tu vrijednost i uzimamo.\\

\underline{Za t = - 52:}\\

Na sličan način kao i na prethodnom primjeru dobijamo vrijednost $s = 7$.\\

Zaključujemo da postoje dva rješenja, te ih uvrštavamo u opšta:\\

\underline{Za $t = - 51 \land s = 5$}\\

$$x = 1, y = 2, z = 3$$
$$Provjera: 1 \cdot 15 + 2 \cdot 33 + 3 \cdot 27 = 162$$\\

\underline{Za $t = - 52 \land s = 7$}

$$x = 5, y = 1, z = 2$$
$$Provjera: 5 \cdot 15 + 1 \cdot 33 + 2 \cdot 27 = 162$$\\

Dakle, postoje dva načina realizacije terapije; prvi način je jedna tableta T1, dvije tablete T2 i 3 tablete T3; drugi način je pet tableta T1, jedna tableta T2 i dvije tablete T3.

\section*{Zadatak 2\label{Z2}}

\hspace{0.65cm}Zadani problem možemo predstaviti u obliku sistema linearnih kongruencija, gdje je $x$ traženi minimalni broj banana:

$$x \equiv 8 \pmod{9} \to NZD(1, 9) = 1$$
$$x \equiv 2 \pmod{10} \to NZD(1, 10) = 1$$
$$x \equiv 0 \pmod{17} \to NZD(1, 17) = 1$$\vspace{1mm}

Dakle, sistem linearnih kongruencija je rješiv što slijedi upravo iz rješivosti svih kongruencija pojedinačno. Rješavamo koristeći kinesku teoremu o ostacima. Najprije provjeramo da li je možemo primjeniti: 

$$NZD(9, 10) = 1$$
$$NZD(9, 17) = 1$$
$$NZD(10, 17) = 1$$\vspace{1mm}

Očigledno je da kinesku teoremu o ostacima možemo primjeniti.

$$n1 \cdot n2 \cdot n3 = 9 \cdot 10 \cdot 17 = 1530$$\vspace{1mm}
$$\lambda 1 = \frac{1530}{9} = 170$$\vspace{1mm}
$$\lambda 2 = \frac{1530}{10} = 153$$\vspace{1mm}
$$\lambda 3 = \frac{1530}{17} = 90$$\vspace{1mm}

Rješenje možemo predstaviti u obliku:

$$x = 170x_{1} + 153x_{2} + 90x_{3} \pmod{1530}$$\vspace{1mm}

Pri čemu su $x_{1}, x_{2}, x_{3}$ ma koja rješenja sistema linearnih kongruencija:

\[
170x_{1} \equiv 8 \pmod{9} \label{eq:kong1} \tag{A}
\]
\[
153x_{2} \equiv 2 \pmod{10} \label{eq:kong2} \tag{B}
\]
$$90x_{3} \equiv 0 \pmod{17}$$

\begin{center}
$x_{3}$ je očigledno bilo koji cijeli broj, dakle $x_{3} = 0$
\end{center}\vspace{1mm}

Kongruencije (\ref{eq:kong1}) i (\ref{eq:kong2}) možemo jednostavno skratiti, te ih izraziti kao diofantove jednačine pa naći potrebnu vrijednost za $x_{1}$ i $x_{2}$:

Prvo skraćujemo kongruencije:

$$(\ref{eq:kong1}) \to 170 > 9 \to mod(170, 9) = 8 \implies 8x_{1} \equiv 8 \pmod{9}$$
$$(\ref{eq:kong2}) \to 153 > 10 \to mod(153, 10) = 3 \implies 3x_{2} \equiv 2 \pmod{10}$$\vspace{1mm}

Odgovarajuće diofantove jednačine:

$$(\ref{eq:kong1}) \to 8x_{1} + 9y = 8 \to NZD(8, 9) = 1, 1 \mid 8$$
$$(\ref{eq:kong2}) \to 3x_{2} + 10y = 2 \to NZD(3, 10) = 1, 1 \mid 2$$\vspace{1mm}

Nalazimo $x_{1}$ i $x_{2}$ tako da $y \in Z$, pri čemu ne moramo rješavati diofantove jednačine, već pogađamo vrijednosti. Dobijamo:

$$x_{1} = 1$$
$$x_{2} = 4$$
\begin{center}
Također $x_{3} = 0$
\end{center}

Pa je opće rješenje:

$$x \equiv 170 \cdot 1 + 153 \cdot 4 + 90 \cdot 0 \pmod{1530}$$
$$x \equiv 782 \pmod{1530}$$

Možemo pisati:

$$x = 782 + 1530t, t\in Z$$\vspace{1mm}

Nalazimo tipično rješenje za koje vrijedi $0 \leq x < 1530$

$$0 \leq 782 + 1530t < 1530$$\vspace{1mm}
$$t \geq - \frac{782}{1530} \>\>\>\> \land \>\>\>\> t < \frac{748}{1530}$$\vspace{1mm}
$$t \geq - 0.51 \>\>\>\> \land \>\>\>\> t < 0.488$$\vspace{1mm}
$$t\in [-0.51, 0.488) \land t \in Z \implies \underline{t = 0}$$\vspace{1mm}

Uvrštavanjem u $x = 782 + 1530t$, se dobije:

$$\underline{x = 782}$$\vspace{1mm}

Zaključujemo da ne samo da je 782 minimalan broj banana potreban da se jednako rasporede u odgovarajuće gomile, već je to i jedini broj za koji može to da se uradi. Provjeriti ćemo rezultat vračajući $x$ u početne jednačine sistema:

$$782 \equiv 8 \pmod{9} \implies 782 + 9y = 8 \implies y \in Z$$
$$782 \equiv 2 \pmod{10} \implies 782 + 10y = 2 \implies y \in Z$$
$$782 \equiv 0 \pmod{17} \implies 782 + 17y = 0 \implies y \in Z$$\vspace{1mm}

Minimalan(i jedini) broj banana potrebnih da bi se jednako raspodijelili je 782.

\section*{Zadatak 4\label{Z4}}

a)
\[
8x + 10y + 17z \equiv 64 \pmod{93} \label{eq:Z4eq1} \tag{1}
\]
\[
12x + 9y + 19z \equiv 3 \pmod{93} \label{eq:Z4eq2} \tag{2}
\]
\[
7x + 14y + 15z \equiv 68 \pmod{93} \label{eq:Z4eq3} \tag{3}
\]
\\
Množimo kongruenciju \ref{eq:Z4eq1} sa 12 i kongruenciju \ref{eq:Z4eq2} sa -8, te ih sabiramo. To ima smisla uraditi jer je $NZD(93, 12) = 1 \land NZD(93, 8) = 1$. Dakle dobijamo:

$$48y + 52z \equiv 744 \pmod{93}$$\vspace{1mm}

Pošto $744 > 93 \implies mod(744, 93) = 0$, kongruencija se svede na:

$$48y + 52z \equiv 0 \pmod{93}$$ \vspace{1mm}

Dalje, množimo kongruenciju \ref{eq:Z4eq2} sa 7 i kongruenciju \ref{eq:Z4eq3} sa -12, te ih sabiramo. NZD u oba slučaja je 1. Dobijamo:

$$105y + 47z \equiv 795 \pmod{93}$$\vspace{1mm}

Daljim skraćivanjem se dobije:

$$12y + 47z \equiv 51 \pmod{93}$$\vspace{1mm}

Sistem smo sveli na sljedeće tri kongruencije:

$$48y + 52z \equiv 0 \pmod{93}$$
$$12y + 47z \equiv 51 \pmod{93}$$
$$7x + 14y + 15z \equiv 68 \pmod{93}$$\vspace{1mm}

Množimo prvu kongruenciju sa -47 i drugu kongruenciju sa 52, sabiramo ih, skratimo, te dobijemo kongruenciju sa jednom nepoznatom:

$$-51y \equiv 48 \pmod{93}$$\vspace{1mm}

Odgovarajuća diofantova jednačina je $-51y + 93k = 48$ gdje je $k$ parametar, $k \in Z$. Pošto je $NZD(93, 51) = 3 \land 3 \mid 48$, diofantova jednačina je rješiva, te očekujemo 3 tipična rješenja. Proširenim euklidovim algoritmom se dobije:

$$1 = 11 \cdot 17 - 6 \cdot 31$$\vspace{1mm}

Interesuje nas rješenje po promjenjivoj $y$:

$$y = -176 + 31t$$\vspace{1mm}

Za tipična rješenja mora vrijediti: $0 \leq y \leq 92 \to t \in [6, 8]$. Dakle dobili smo 3 tipična rješenja koja glase:

$$y = 10, y = 41, y = 72$$\vspace{1mm}

Za $y = 10$ kongruencija ima najmanje tipično rješenje, pa opće rješenje možemo pisati u obliku $y \equiv 10 \pmod{31}$. Da ne bi razmatrali svaki od tipičnih rješenja zasebno, možemo na sljedeći način napisati opće rješenje:

$$y = 10 + 31t, t \in Z$$\vspace{1mm}

Dobiveno opće rješenje vraćamo u prvu kongruenciju:

$$48(10 + 31t) + 52z = 0 \pmod {93}$$\vspace{1mm}

Tj. skraćivanjem:

$$52z = -15 \pmod{93}$$\vspace{1mm}

Pa je odgovarajuća diofantova jednačina $52z\> +\> 93k = -15, k \in Z$. $NZD(93, 520) = 1 \> \land \> 1 \mid 15$, dakle diofantova jednačina je rješiva te očekujemo jedinstveno tipično rješenje. Dobije se $z = -510 + 93t, t \in Z$. Pa je tipično rješenje:

$$z = 48 \to z \equiv 48 \pmod{93}$$\vspace{1mm}

Uvrštavamo $z = 48$ i $y = 10 + 31t, t \in Z$ u kongruenciju \ref{eq:Z4eq3}. Dobije se:

$$7x = -48 - 62t \pmod{93}, t \in Z$$\vspace{1mm}

Odgovarajuća diofantova jednačina: $7x + 93k = -48 - 62t, t, k \in Z$. Diofantova jednačina je rješiva, te očekujemo jedno tipično rješenje za svaki cijeli broj t, $NZD(93, 7) = 1 \land 1 \mid -48 - 62t$.
Dobije se $x = -672 -868t + 93s, t, s \in Z$. Pa je $s = 8 + \frac{868}{93} \cdot t, t \in Z$. Tipično rješenje je jedinstveno i ono glasi:

$$x = 72 - 868t, t \in Z \to x \equiv 72 - 31t \pmod{93}$$\vspace{1mm}

Dakle, rješenja sistema su:

$$x \equiv 72 - 31t \pmod{93}, t \in Z$$
$$y \equiv 10 \pmod{31}$$
$$z \equiv 48 \pmod{93}$$\vspace{1mm}

Pri čemu svako tipično rješenje koje smo dobili za $y$ odgovara da bude rješenje sistema. Dakle, ovaj sistem ima 3 tipična rješenja:

$$x = 310, y = 10, z = 48$$ 
$$x = 1271, y = 41, z = 48$$
$$x = 2232, y = 72, z = 48$$\vspace{1mm}
\\
b)
\[
24x + 27y \equiv 9 \pmod{78} \label{eq:Z4eq1b} \tag{1}
\]
\[
10x + 12y \equiv 16 \pmod{78} \label{eq:Z4eq2b} \tag{2}
\]
\\

Ne možemo množiti kongruencije odgovarajućim brojevima jer njihovi odgovarajući $NZD \neq 1$. Dakle, moramo postepeno smanjivati koeficijent uz neku nepoznatu u nekoj kongruenciji, dok ne nestane potpuno. Uradit ćemo sljedeće korake, kako bi nepoznatu $x$ izbacili iz druge kongruencije:\vspace{1mm}

\begin{center}
1.) Množimo kongruenciju \ref{eq:Z4eq2b} sa -1 i dodajemo kongruenciji \ref{eq:Z4eq1b}. Ovaj korak uradimo 2 puta uzastopno.


2.) Množimo kongruenciju \ref{eq:Z4eq1b} sa -1 i dodajemo kongruenciji \ref{eq:Z4eq2b}. Ovaj korak uradimo 2 puta uzastopno također.


3.) Uradimo 1. ponovno, ali ovaj put samo jednom.


4.) Uradimo 2. ponovno, ali ovaj put samo jednom.
\end{center}

Dobili smo sistem:

$$2x - 3y \equiv -85 \pmod{78}$$
$$9y \equiv 147 \pmod{78}$$\vspace{1mm}

Sistem sa jednom nepoznatom svodimo na diofantovu jednačinu $9y + 78k = 147$, gdje je k parametar. $NZD(78, 9) = 3 \land 3 \mid 69$. Zaključujemo da je diofantova jednačina rješiva, i očekujemo 3 tipična rješenja. Podijelimo diofantovu jednačinu sa 3, dobijamo $3y + 26k = 23$. Proširenim euklidovim algoritmom dobijemo $1 = 9 \cdot 3 - 1 \cdot 26$. Pa je $y = 207 + 26t, t \in Z$. Za $t = -7, t = -6, t = -5$ dobijamo tipična rješenja ove kongruencije:

$$y = 25, y = 51, y = 77$$\vspace{1mm}

Najmanje tipično rješenje je y = 25, pa možemo također pisati:

$$y \equiv 25 \pmod{26}$$ \vspace{1mm}

Da ne bi morali za svako tipično rješenje računati sistem, pišemo općenito $y = 25 + 26t, t \in Z$. Isti izraz vraćamo u prvu kongruenciju tj.

$$2x - 3y \equiv -85 \pmod{78} \to 2x \equiv -10 + 78t \pmod{78}$$
$$2x \equiv -10 \pmod{78}$$\vspace{1mm}

Odgovarajuća diofantova jednačina je $2x + 78k = -10$, gdje je k parametar. $NZD(78, 2) = 2 \land 2 \mid 10$, dakle diofantova jednačina je rješiva i očekujemo 2 tipična rješenja. Rješenje diofantove jednačine je $x = -5 + 39t, t \in Z$. Iz rješenja slijedi da za $t = 1 \land t = 2$ imamo tipična rješenja:

$$x = 34, x = 73$$\vspace{1mm}

Najmanje tipično rješenje je $x = 34$, pa možemo također pisati:

$$x \equiv 34 \pmod{39}$$\vspace{1mm}

Zaključujemo da je rješenje sistema:

$$x \equiv 34 \pmod{39}$$
$$y \equiv 25 \pmod{26}$$\vspace{1mm}

Ili zapisano u vidu tipičnih rješenja; ovaj sistem ima 6 tipičnih rješenja:

$$x = 34, y = 25; x = 34, y = 51; x = 34, y = 77$$
$$x = 73, y = 25; x = 73, y = 51; x = 73, y = 77$$

\end{document}